\documentclass{article}

\input{Packages}
\usepackage[backend=bibtex, sorting=none]{biblatex}
\addbibresource{bibliography.bib}


\title{Project 1: Empirical Observation of the }
\author{Claudio Vestini}


\begin{document}

\maketitle

\section{Introduction \& Satellite Selection}

Since the first successful deployment of a human-made object into Earth's orbit with \textit{Sputnik 1} in 1957, over 15,000 satellites have been placed in orbit around our planet~\cite{lookup-lepoint2025}. Of these, more than 13,000 remain operational today, and this unprecedented active percentage is continuously growing. The vast majority are of American origin (roughly \SI{74}{\percent}), and most belong to SpaceX's Starlink constellation, which alone accounts for approximately \SI{86}{\percent} of all U.S. satellites and \SI{64}{\percent} of the world's total active population. Engineered to provide high-speed, low-latency internet connectivity to underserved rural areas of the world for a moderate service price, Starlink has been rapidly expanding its constellation, with thousands of new satellites being launched every year through proprietary SpaceX vehicles.

\begin{figure}[h!]
    \centering
    \includegraphics[width=\textwidth]{LaTeX/Figures/Starlink Constellation.png}
    \caption{Starlink constellation as of October 19th 2025, 16:59:33. Each red square represents an active satellite. Highlighted in yellow is the orbital track of selected satellite \textit{Starlink-3988}, with relevant orbital information displayed. The image is a screenshot from the Starlink Map website~\cite{starlinkmap.org}.}
    \label{fig:constellation}
\end{figure}

These satellites have already been used for several key applications, including enabling WiFi connectivity in commercial airlines and cruises, providing life-critical broadband in hurricane-ravaged coastal towns and earthquake-shaken regions, and facilitating allied command and control operations in the Russo-Ukrainian War. The constellation includes nearly 9,700 satellites (as of the writing of this report), and is visualised in Figure~\ref{fig:constellation}. Each unit is equipped with four beamforming, phased array antennas, each of which electronically steers the \SI{11.7}{\giga\hertz} collimated downlink signals in real time to a \SI{24}{\kilo\metre}-diameter ground coverage cell that can serve up to 8,000 simultaneous customers. Furthermore, the satellites communicate with one another via five on-board optical lasers at vacuum light-speed, enabling lower effective latencies between far-away cities compared to underwater cables. This is particularly relevant for applications in stock market trading, where saving a few milliseconds in latency can have a huge impact on the decision-making reactions to market fluctuations. To add to the list of groundbreaking engineering innovations that SpaceX developed for Starlink satellites, the orbital insertion procedure is entirely novel: the satellites are deployed in clusters of up to 23 units\footnote{For the larger v2 model, down from up to 60 units of the smaller v1/v1.5 model} at an altitude of roughly \SI{280}{\kilo\metre}, then their orbits are slowly raised to around \SI{550}{\kilo\metre} in two stages using Kripton gas ion thrusters\footnote{This is a more cost-efficient alternative to the customary option, Xenon gas}. This unusual manoeuvre leaves the satellites bunched up in characteristic lines for a long period of time before successful insertion, and these can be observed from the surface of the Earth as shown in Figure~\ref{}.

This project aims to investigate the orbit of an active, Earth-orbiting satellite through empirical observation and data collection.







\section{Data Collection}

\section{Estimate of Period and Angular Velocity}

\section{Estimate of Semi-Major Axis and Position Prediction}

\section{Estimate of Orbital Elements}

\section{Error Analysis}

\section{Conclusion}

\section{Visualisation}

\printbibliography

\end{document}
