\documentclass{article}

\input{Packages}
\usepackage[backend=bibtex, sorting=none]{biblatex}
\addbibresource{bibliography.bib}


\title{Project 1: Empirical Observation of the }
\author{Claudio Vestini}


\begin{document}

\maketitle

\section{Introduction \& Satellite Selection}

Since the first successful deployment of a human-made object into Earth's orbit with \textit{Sputnik 1} in 1957, over 15,000 satellites have been placed in orbit around our planet~\cite{lookup-lepoint2025}. Of these, more than 13,000 remain operational today, and this unprecedented active percentage is continuously growing. The vast majority are of American origin (roughly \SI{74}{\percent}), and most belong to SpaceX's Starlink constellation, which alone accounts for approximately \SI{86}{\percent} of all U.S. satellites and \SI{64}{\percent} of the world's total active population.

Engineered to provide high-speed, low-latency internet connectivity to underserved rural areas of the world for a moderate service price, Starlink has been rapidly expanding its constellation, with thousands of new satellites being launched every year through proprietary SpaceX vehicles. 

This project aims to investigate the orbit of an active, Earth-orbiting satellite through empirical observation and data collection.

\section{Data Collection}

\section{Estimate of Period and Angular Velocity}

\section{Estimate of Semi-Major Axis and Position Prediction}

\section{Estimate of Orbital Elements}

\section{Error Analysis}

\section{Conclusion}

\section{Visualisation}

\printbibliography

\end{document}
